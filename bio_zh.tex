%; /usr/bin/env mkdir -p .build && xelatex -output-directory=.build -halt-on-error $0 && cp .build/*.pdf ./; exit

% Add 'print' as an option in the square bracket to remove colors from this template for printing
\documentclass[]{friggeri-cv}
\begin{document}

\header{陈琳}{}{}

% ----------------------------------------------------------------------------------------
% SIDEBAR SECTION
% ----------------------------------------------------------------------------------------

\begin{aside} % In the aside, each new line forces a line break
  \section{联系方式}
  +1 (650) 307-5688
  \href{mailto:clphoenix37+cv@gmail.com}{clphoenix37@gmail.com}
  \href{https://clphoenix.wordpress.com/}{clphoenix.wordpress.com}
\end{aside}

陈琳喜欢做对人类社会有意义的事。
她想促进人类的进步。
她热衷于自我提升,并通过以身作则来影响周围的人。
在经过了虚无,享乐,实用主义之后,她认为人不是为自己,而是为人类的发展而活的。
而一个人的人格,她认为是最重要的。
做一个更好的人,和帮助别人做一个更好的人,是她唯一的动机。

% ----------------------------------------------------------------------------------------
% EXPERIENCE SECTION
% ----------------------------------------------------------------------------------------

\section{经历}
在美国的一所较高端的养老院做过膳食辅助的工作,和老人聊天,对人的一生有颇多感悟。

大学毕业后不清楚自己想做什么工作。
于是博览群书,旅居,研究了几年投资,哲学,名人传记。
由此对人,时间,和空间有了更多认识,看清了人生的意义和方向。
在心理足够成熟后,决定生养下一代,并用心教育。

大学时做过一些志愿者的工作(世界音乐教育大会,贫困小学支教,慈善义卖,
无家可归者的庇护所,动物庇护所),团委志工部,学生会权益部,班级生活委员。
编导,制片了一个40人参演的MV。
在美国大学餐厅打过工,和同学合作很愉快。在乐队里拉小提琴,设计乐队海报。

% ----------------------------------------------------------------------------------------
% EDUCATION SECTION
% ----------------------------------------------------------------------------------------

\section{教育}

\begin{entrylist}
  % ------------------------------------------------
  \entry{2012 --- 2014}
  {应用数学 文理学士}
  {伊利诺伊大学香槟分校}
  {}
  % ------------------------------------------------
  \entry{2009 --- 2011}
  {应用数学}
  {北京科技大学}
  {}
  % ------------------------------------------------
\end{entrylist}

\end{document}
